\chapter{Proxy RouteFlow em Java}

\section{Introdução aos Proxies RouteFlow (RFProxy)}

Nos capítulos à respeito do projeto RouteFlow foi dado
uma visão geral de todos os três módulos usados para
construir o projeto. Esse capítulo se foca principalmente no 
módulo proxy, desenvolvido como foco principal do trabalho de
conclusão de curso.

Como já dito nos capítulos anteriores, o proxy RouteFlow ou
RFProxy atua como um tradutor de protocolos, convertendo
mensagens no protocolo RouteFlow para o protocolo OpenFlow
e vice-versa. Todos os eventos do ambiente físico são notificados
ao servidor na forma de mensagens, sendo as mesmas tratadas 
e processadas de acordo com cada tipo de evento. Cada evento gera uma 
ação ao servidor RouteFlow, que responde na forma de uma 
mensagem ao proxy RouteFlow, instruindo-o a tomar certa atitude.
Dentro dessas atitudes estão a criação das regras nos \textit{
switches OpenFlow}, fazendo toda a rede física se comportar
da maneira proposta pelo ambiente virtual e seus algoritmos de
roteamento.

As mensagens do protocolo RouteFlow, como já citado nos
capítulos anteriores, vem via um mecanismo de troca de mensagens
entre processos (IPC) implementado através do banco de dados
centralizado. Tal característica força o desenvolvedor a escolher
as linguagens de programação com suporte à manipulação do
banco de dados usado pelo projeto RouteFlow, que no caso
é o MongoDB.

As mensagens do protocolo \textit{OpenFlow} são advindas do ambiente
físico e necessitam de algum mecanismo de captura e processamento.
Tal mecanismo é provido pelos softwares de controle \textit{
OpenFlow}. Dentre os diversos disponíveis, os mais famosos
são o NOX, o POX e o Floodlight. Todos os controladores
\textit{OpenFlow} são operados apenas como 
interfaces de programação (APIs), provendo todas as funções de captura
de eventos e criação de mensagens no protocolo \textit{OpenFlow}.

Cada um dos controladores citados acima foi desenvolvido em
uma linguagem de programação diferente, forçando o desenvolvedor
a trabalhar na mesma linguagem do controlador de forma a 
opera-lo como forma de interface de programação (API). 
Cada controlador possui vantagens exclusivas, tendo sido construídos
com propósitos diferentes. O controlador NOX se destaca pela
performance, pelo fato de ser construído em C++ e possuir
código compilado. O controlador POX se destaca pela
facilidade de programação provida pela linguagem Python mas
possui desempenho inferior devido ao fato do Python ser
interpretado. 

O controlador escolhido para uso neste trabalho foi 
o Floodlight. Sendo desenvolvido para ser usado
principalmente por  administradores de rede, possui uma 
série de aplicações nativas que permitem o seu controle
via aplicações externas. Uma dessas aplicações nativas 
permite a recepção de mensagens via mensagens REST,
possibilitando ao desenvolvedor manipular sua rede sem
a necessidade de programar um módulo dedicado. Tal fato
dificultou o desenvolvimento do trabalho, visto que grande 
parte da documentação disponível para consulta era voltada
apenas para os administradores, sem levar em consideração os 
desenvolvedores de aplicações em Java. O fato
de ter sido construído em Java pode impactar em seu desempenho
mas tal impacto é recompensado visto o grande número de 
recursos disponíveis. Um dos recursos que pode ser citado
é a interface gráfica usada para visualização do estado da rede que
no futuro poderá ser adaptada para uso como o projeto 
RouteFlow. A comunidade de usuários é grande e bastante
ativa, fato notado pela grande quantidade de postagens no
fórum oficial do controlador e pela disposição dos principais
desenvolvedores na solução de problemas encontrados ou
dúvidas que surgiram ao longo do desenvolvimento.

Os dois proxies RouteFlow já existentes durante a 
criação desse trabalho seguem um mesmo algoritmo, o que 
facilita muito o entendimento geral. A maior dificuldade no
desenvolvimento de um novo proxy RouteFlow são
as características exclusivas de cada linguagem de programação,
cabendo ao desenvolvedor as adaptações necessários para 
implementar o algoritmo sem grandes alterações
na ideia básica. A linguagem Java já possuía uma
ótima API para comunicação com o banco de dados MongoDB
e para a manipulação de mensagens no protocolo JSON. Um fato
que trouxe certa dificuldade no trabalho foi o desenvolvimento
de uma aplicação com múltiplas threads em Java. 

A Figura \ref{fig:esquematicoProxy} ilustra de forma simples as
funções básicas bem como a direção das informações tratadas
pelo proxy RouteFlow. É interessante notar que o 
proxy só se comunica com o servidor RouteFlow através
do mecanismo de troca mensagens entre processos (IPC) baseado
no banco de dados centralizado.
\newline
\newline
\newline
\newline


\begin{figure}[h] 
\centering
\includegraphics[width=120mm]{esquematico_geral_proxy.png}
\caption{Esquema Geral do Proxy RouteFlow (RFProxy).}
\label{fig:esquematicoProxy} 
\end{figure}
 

O próximo capitulo irá detalhas as funções internas e principais
estruturas do proxy RouteFlow.

\section{Descrição Geral da Estrutura do Proxy RouteFlow em Java}

Todos os componentes do proxy RouteFlow em Java foram agrupados em classes.
O agrupamento em classes facilitou a 
organização geral do código bem como a sua futura manutenção. 
Abaixo temos os principais componentes do proxy como 
suas respectivas descrições:

\begin{itemize}
\item \textit{MongoIPCMessageService:} Responsável pela
 comunicação entre o servidor RouteFlow e o proxy 
RouteFlow. A arquitetura básica do projeto RouteFlow 
faz uso de um banco de dados Não SQL para troca de 
mensagens entre seus componentes, sendo que o banco 
de dados escolhido foi o MongoDB. 

O MongoDB possui 
alto desempenho sendo totalmente escrito em C++, outro
 aspecto importante é o fato de não ser SQL, o 
que facilita a sua integração com as  principais linguagens
 de programação. 
O RouteFlow cria inúmeras tabelas no banco de dados, cada uma 
responsável pela comunicação entre um par 
de componentes, como toda comunicação é feita através de um 
sistema de banco de dados é possível que a 
comunicação entre componentes seja feita de forma simples, sem 
nenhum vinculo com a linguagem de 
implementação do mesmo. 

O componente MongoIPCMessageService 
cria um mecanismo de troca de mensagens entre processos (IPC) 
entre o servidor RouteFlow e o proxy RouteFlow. Os comandos são 
enviados de um componente para o outro 
na forma de mensagens pré-definidas. Cada mensagem define uma 
ação à ser tomada em relação aos eventos 
que vão ocorrendo ao longo da execução do ambiente físico e virtual. No corpo da 
mensagem estão os parâmetros que 
deverão ser usados para tomada da ação. Todas as mensagens que 
são colocadas na tabela pelo servidor RouteFlow 
possuem um campo que indica se a mesma já foi tratada e em caso 
negativo cabe ao proxy tomar a ação e 
atualizar o campo da mensagem. Para tratamento das mensagens é 
gerado uma thread em looping infinito 
cujo único proposito de existência é o tratamento de novas mensagens. 
Essa característica pretende ser melhorada nas próximas versões 
do RouteFlow;
\item \textit{RFProtocolFactory:} Responsável pela criação das 
mensagens do protocolo RouteFlow. Cada 
tipo de mensagem RouteFlow é representada por um código e 
por uma respectiva classe. É papel do 
RFProtocolFactory retornar os objetos de mensagens à partir de 
seu código. Esta estrutura facilita a 
inserção de novas mensagens no ambiente evitando a reprogramação 
de outras classes;
\item \textit{RFProtocolProcessor:} Responsável pelo processamento 
de mensagens vindas do servidor 
RouteFlow. Este componente define como será tratada cada mensagem 
vinda do servidor RouteFlow. As 
mensagens são lidas através do IPC e repassadas para tratamento.
\item \textit{AssociationTable:} Tabela mantida pelo proxy para armazenar 
a associação entre portas no 
ambiente virtual e portas no ambiente físico. A Tabela \ref{tab:tabela_associacao} nos 
mostra a estrutura básica da tabela de associação:

\begin{table}[h]
\centering
\begin{tabular}{|l|c|c|}
\hline
Tipo & Função\\
\hline
\hline
$dp_-id,\ dp_-port,\ vs_-id,\ vs_-port$ & Interface física com interface virtual\\
\hline
$vs_-id,\ vs_-port,\ dp_-id,\ dp_-port$ & Interface virtual com interface física\\
\hline
\end{tabular}
\caption{Representação da tabela de associação.}
\label{tab:tabela_associacao}
\end{table}

Repare que a Tabela \ref{tab:tabela_associacao} possui redundância
em suas colunas. Tal fato exige uma dupla atualização de campos
sempre que ocorre uma nova associação. Isso será resolvido
nas próximas versões do proxy.

\end{itemize} 

No Algoritmo abaixo temos uma visão geral da ideia usada para
construção do proxy, lembrando que a orientação a objetos
da linguagem Java força o código executável a ser construído
no formato de classe. Para deixar o código mais limpo, todas
as características especificas da linguagem Java foram omitidas,
deixando o código no formato de pseudo algoritmo.
\newline

%\begin{algorithmic}
\noindent
\tcc*[h]{Declaração dos componentes básicos.}
\newline

\noindent
\tcc*[h]{Provedor de serviços OpenFlow, fornecido pelo
controlador.}
\newline
\textit{floodlightProvider} $\leftarrow$ \textbf{IFloodlightProviderService};
\newline

\noindent
\tcc*[h]{Gerador de logs.}
\newline
\textit{logger} $\leftarrow$ \textbf{Logger};
\newline

\noindent
\tcc*[h]{Criação do mecanismo de comunicação
 entre processos, será explicado com mais detalhes.}
\newline
\textit{ipc} $\leftarrow$ \textbf{MongoIPCMessageService};
\newline

\noindent
\tcc*[h]{Gerador de mensagens RouteFlow. Cada mensagem RouteFlow
tem seu tipo e seus próprios parâmetros. Todas as mensagens
são representadas na forma de classes, a função em questão
basicamente retorna a classe representante da função de acordo
com o seu tipo.}
\newline
\textit{factory} $\leftarrow$ \textbf{RFProtocolFactory};
\newline

\noindent
\tcc*[h]{Processador de mensagens RouteFlow. Sera explicado
com mais detalhes.}
\newline
\textit{processor} $\leftarrow$ \textbf{RFProtocolProcessor};
\newline

\noindent
\tcc*[h]{Tabela de associação.}
\newline
\textit{table} $\leftarrow$ \textbf{AssociationTable};
\newline

\noindent
\tcc*[h]{Função usada para configurar regras, parcialmente fornecida
pelo controlador.}
\newline
\textit{\textbf{flowConfig()}};
\newline

\noindent
\tcc*[h]{Função usada para deletar regras, parcialmente fornecida
pelo controlador.}
\newline
\textit{\textbf{flowDelete()}};
\newline

\noindent
\tcc*[h]{Função usada para adicionar regras, parcialmente fornecida
pelo controlador.}
\newline
\textit{\textbf{flowAdd()}};
\newline

\noindent
\tcc*[h]{Função usada para registrar o proxy no
controlador Floodlight. Burocracia exigida pelo controlador.}
\newline
\textit{\textbf{init()}};
\newline

\noindent
\tcc*[h]{Função usada para o processamento dos pacotes
de entrada OpenFlow.}
\newline
\textit{\textbf{processPacketInMessage()}};
\newline

\noindent
\tcc*[h]{Função usada para iniciar a execução do proxy.
Burocracia exigida pelo controlador.}
\newline
\textit{\textbf{startUp()}};
\newline

\noindent
\tcc*[h]{Função usada para definir que tipo de mensagens 
OpenFlow serão tratadas pelo algoritmo do proxy e especificamente
por quais funções. O proxy habilita a função \textit{processPacketInMessage}
a tratar as mensagens de entrada OpenFlow.}
\newline
\textit{\textbf{receive()}};
\newline

\noindent
\tcc*[h]{Função executada sempre que um switch OpenFlow ingressa
na rede. Sempre que isso acontece é necessário informar
ao servidor RouteFlow as informações básicas do ingressante
para que o mesmo o associe a uma máquina virtual do
ambiente virtual.}
\newline
\textit{\textbf{addedSwitch()}};
\newline

\noindent
\tcc*[h]{Função executada sempre que um switch OpenFlow
abandona a rede. Sempre que isso acontece é necessário informar
ao servidor RouteFlow para o mesmo remova o registro e a
associação com a máquina virtual do ambiente virtual.}
\newline
\textit{\textbf{removedSwitch()}};
\newline

%\SetAlgoLined
%\Left$\leftarrow$ \FindCompress{$Im[i,j-1]$}\;
%\While{not at end of this document}{
%read current\;
%\eIf{understand}{
%go to next section\;
%current section becomes this one\;}
%{go back to the beginning of current section\;}
%}
%\caption{Ideia Básica do RFProxy}
%\label{alg:basico}
%\end{algorithmic}

Esse é o algoritmo básico do proxy RouteFlow. Todos os outros 
proxies implementados anteriormente seguem essa mesma
arquitetura.

Abaixo temos um pseudo algoritmo do mecanismo de comunicação
entre processos usado para funcionamento do proxy, no algoritmo
anterior a classe é chamada de \textbf{MongoIPCMessageService}:
\newline

\noindent
\tcc*[h]{Definição dos parâmetros de acesso ao
 banco de dados como nome da coleção de dados, nome
do banco de dados e endereço do banco de dados.}
\newline

\noindent
\tcc*[h]{Função usada para criação da thread não bloqueante
que executará a função ListenWorker. A função ListenWorker
será mostrada logo abaixo.}
\newline
\textit{\textbf{listen() \{}}
\newline
\tcc*[h]{Cria a nova thread, associa a função listenWorker e
inicia a sua execução.}
\newline
\textit{\textbf{\};}}
\newline

\noindent
\tcc*[h]{Função usada para checagem constante do banco
de dados à procura de alguma mensagem destinada
ao processo registrado. O processo se registra através da
criação de um canal de comunicação. O canal de comunicação
nada mais é que uma coleção de dados no banco de dados.}
\newline
\textit{\textbf{listenWorker() \{}}
\newline
\newline
\tcc*[h]{Define os parâmetros de busca. Os principais foram
definidos no início do algoritmo. É definido o nome da coleção
que será checada de tempos em tempos em busca de novas
mensagens.}
\newline
\newline
\indent
\textit{\textbf{enquanto(verdadeiro) \{}}
\newline
\indent
\indent
\textit{\textbf{se(mensagem não lida) \{}}
\newline
\indent
\indent
\indent
\tcc*[h]{Extrai os dados importantes e altera o parâmetro lido
do banco de dados para verdadeiro. Os dados são passados
para a função processor, mostrada no primeiro algoritmo.}
\newline
\indent
\indent
\textit{\textbf{\};}}
\newline
\indent
\indent
\textit{\textbf{senao \{}}
\newline
\indent
\indent
\indent
durma
\newline
\indent
\indent
\textit{\textbf{\};}}
\newline
\indent
\textit{\textbf{\};}}
\newline
\textit{\textbf{\};}}
\newline

\noindent
\tcc*[h]{Função usada para envio de dados ao banco de dados.
O termo envio significa inserir um novo conjunto de dados
na coleção. Serve basicamente para envio de mensagens à outros
processos.}
\newline
\textit{\textbf{send() \{}}
\newline
\textit{\textbf{\};}}
\newline

\noindent
\tcc*[h]{Função usada para criar um novo canal de comunicação.
Usada na primeira vez que o mecanismo de comunicação é
executado.}
\newline
\textit{\textbf{createChannel() \{}}
\newline
\textit{\textbf{\};}}
\newline

Este é basicamente a estrutura básica do algoritmo usado
para criação do mecanismo de comunicação entre processos.
A função listenWorker, sempre que nota o recebimento de
uma nova mensagem, encapsula os dados e transfere a função 
processor. A função processor, devido a orientação à objetos,
é sobrecarregada na definição do código do proxy. Assim é usada
para tomar as ações necessárias ao funcionamento do sistema.
Abaixo temos um pseudo código que será mais explicado a seguir:
\newline
\newline
\noindent
\tcc*[h]{Função usada para tratamento das mensagens
recebidas através do mecanismo de comunicação entre 
processos.}
\newline
\textit{\textbf{processor(dados) \{}}
\newline
\newline
\indent
\tcc*[h]{O parâmetro dados constitui basicamente do tipo
de mensagem e seus respectivos campos. Para cada tipo
de mensagem é necessário tomar uma ação.}
\newline
\indent
\textit{\textbf{caso(tipo) \{}}
\newline
\indent
\indent
tipo 1: \tcc*[h]{ação 1}
\newline
\indent
\indent
tipo 2: \tcc*[h]{ação 2}
\newline
\indent
\indent
...
\newline
\indent
\indent
tipo N: \tcc*[h]{ação N}
\newline
\indent
\textit{\textbf{\};}}
\newline
\textit{\textbf{\};}}
\newline

Esse é basicamente o algoritmo usado em todos os proxies RouteFlow.
As principais diferenças são relacionadas às linguagens de programação
 e ao controlador usado como interface com o protocolo
\textit{OpenFlow}.

\section{Descrição das Mensagens Traduzidas pelo Proxy RouteFlow}

Todas as mensagens mostradas na lista abaixo são representadas
via classes na linguagem Java. As classes foram geradas
automaticamente por um script em Python. Originalmente
o script só fazia a geração nas linguagens C++ e Python,
para que fossem utilizadas respectivamente pelos controladores
NOX e POX. Foi fruto desse trabalho de conclusão de curso
expandir o suporte do script para a linguagem Java. O script
é construído em Python e recebe como entrada uma lista com
a definição de campos e tipos de cada mensagem e retorna
os arquivos com as respectivas implementações. 

A Figura \ref{fig:mensagens}
ilustra o sentido de cada mensagem, nos dizendo que tipo de mensagem
é enviada para cada módulo. Vale lembra que as mensagens
são enviadas ao banco de dados centralizado para atuar no
mecanismo de troca de mensagens entre processos (IPC).
\newline

\begin{figure}[h] 
\centering
\includegraphics[width=140mm]{mensagens.png}
\caption{Fluxo das mensagens.}
\label{fig:mensagens} 
\end{figure}

\begin{itemize}
\item \textit{PortRegister:} mensagem utilizada sempre que um
novo \textit{switch} OpenFlow ingressa no ambiente físico. As mensagens 
servem para informar ao servidor RouteFlow sobre a disponibilidade
de portas para que as mesmas sejam associadas às interfaces
das máquinas virtuais no ambiente virtual.
\item \textit{PortConfig:} mensagem utilizada solicitação de 
configuração de alguma porta do ambiente físico pelo servidor
RouteFlow. Atualmente não é utilizada.
\item \textit{DatapathConfig:} mensagem utilizada para
configuração dos \textit{switches} via protocolo OpenFlow. A 
configuração envolve aspectos sobre como tratar certos tipos
de pacotes baseado em seus protocolos.
\item \textit{RouteInfo:} mensagem não utilizada.
\item \textit{FlowMod:} mensagem utilizada para 
solicitação da instalação de regras nos switches 
OpenFlow. As mensagens FlowMod possuem em seu 
corpo um conjunto de parâmetros que define uma nova regra 
a ser aplicada à um \textit{switch} OpenFlow. Cabe ao proxy 
criar a regra e envia-la corretamente ao \textit{switch}. 
Para envio das regras é necessário a manipulação do 
protocolo OpenFlow. O Floodlight fornece uma série de 
funções para esse proposito, sendo usado como uma 
API de comunicação entre os \textit{switches} OpenFlow e o 
proxy.
\item \textit{DatapathPortRegister:} mensagem utilizada 
para registrar as portas dos \textit{switches} OpenFlow 
no servidor RouteFlow. Esse registro é feito para que cada 
porta seja associada à uma porta da máquina 
virtual do ambiente virtual.
\item \textit{DatapathDown:} mensagem utilizada para que 
o proxy informe ao servidor RouteFlow sobre a 
desconexão de um \textit{switch} OpenFlow. O Floodlight, no papel 
de controlador OpenFlow, mantem um conjunto de 
informações à respeito dos \textit{switches} OpenFlow ativos, 
podendo detectar quedas nas conexões dos mesmos. O 
servidor RouteFlow necessita ter esse tipo de informação para 
possíveis alterações nas regras dos 
\textit{switches} OpenFlow.
\item \textit{VirtualPlaneMap:} mensagem utilizada para que 
o proxy associa cada porta do \textit{switch} 
OpenFlow à uma porta de uma máquina virtual no ambiente 
virtual.
\item \textit{DataPlaneMap:} mensagem utilizada para que o 
servidor RouteFlow informe ao proxy à 
respeito de uma associação de porta bem sucedida. O proxy 
mantém uma tabela de associação entre portas 
no ambiente virtual e portas no ambiente físico.
\end{itemize}

