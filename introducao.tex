\chapter{Introdução}

\section{Objetivo do Trabalho}
O trabalho tem como objetivo principal agregar uma
 nova funcionalidade ao Projeto \textit{RouteFlow}. A 
funcionalidade será o suporte nativo ao controlador 
\textit{Floodlight}. Os controladores são usados 
pelo 
\textit{RouteFlow} como uma interface de comunicação
 entre os softwares de roteamento e os \textit{switches} 
\textit{Openflow}. Cada controlador possui certas
 características juntamente com recursos exclusivos e 
com isso esperá-se que o \textit{RouteFlow} agregue
 as melhores ferramentas disponíveis no controlador 
\textit{Floodlight}.
 A implementação atual do RouteFlow possui suporte
 aos controladores \textit{NOX} e \textit{POX}, sendo 
desenvolvidos respectivamente em C++ e Python. O
 Floodlight foi totalmente desenvolvido em \textit{Java} 
tendo suporte ao estilo de comunicação distribuída
 \textit{REST (Representational State Transfer)}, 
sendo possível utilizado sem a necessidade de programação,
  através de mensagens \textit{REST}.

\section{Contribuições}
Como principal contribuição do trabalho podemos citar
 a integração que haverá entre o \textit{RouteFlow} 
e a comunidade de usuários do controlador \textit{Floodlight}. 
A comunidade poderá realizar simulações 
ou até experimentos em ambientes reais contribuindo ainda
 mais com o avanço do \textit{RouteFlow}. O uso
constante do controlador \textit{Floodlight} pelo
 \textit{RouteFlow} servirá como uma plataforma de testes
para o próprio controlador, contribuindo para a localização de
 possíveis erros ou até mesmo falta de estabilidade.
Outra 
contribuição importante que pode ser citada é a absorção pelo
  \textit{RouteFlow} das melhores 
ferramentas providas pelo \textit{Floodlight}, tornando-o
 cada vez mais completo.  
